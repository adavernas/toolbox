\documentclass[12pt,english]{article}

\usepackage[toc,page]{appendix}
\usepackage{rotfloat}

\usepackage{amsthm,amsmath,amssymb,latexsym}

\usepackage{babel}
\DeclareFontFamily{U}{mathx}{\hyphenchar\font45}
\DeclareFontShape{U}{mathx}{m}{n}{<-> mathx10}{}
\DeclareSymbolFont{mathx}{U}{mathx}{m}{n}
\DeclareMathAccent{\widebar}{0}{mathx}{"73}
\usepackage{graphicx}
\usepackage[space]{grffile}
\usepackage{setspace}
\usepackage{subcaption}
\usepackage{fullpage}
\usepackage{booktabs}
\usepackage{fancyhdr}
\usepackage{pdfpages}
\usepackage[flushleft]{threeparttable}
\usepackage{enumerate}
\usepackage{rotfloat}
\usepackage{epstopdf}
\usepackage{bbm}
\usepackage{geometry}
\usepackage{pdflscape}
\usepackage{ragged2e}
\usepackage[hyperfootnotes=false,hidelinks]{hyperref}
\usepackage{color}
\usepackage{endnotes}
\onehalfspacing
\usepackage{tabularx, booktabs}
\usepackage{color}
\usepackage{standalone}
\usepackage{bibentry} 
\usepackage{afterpage}
%\allowdisplaybreaks
\usepackage[round]{natbib}
\bibliographystyle{abbrvnat}
\setcitestyle{authoryear,open={(},close={)}}

\usepackage{array,multirow}
\usepackage{rotating}

\newcolumntype{Y}{>{\centering\arraybackslash}X}

\usepackage[font=footnotesize,labelfont=bf]{caption}

\DeclareFontFamily{U}{BOONDOX-calo}{\skewchar\font=45 }
\DeclareFontShape{U}{BOONDOX-calo}{m}{n}{
  <-> s*[1.05] BOONDOX-r-calo}{}
\DeclareFontShape{U}{BOONDOX-calo}{b}{n}{
  <-> s*[1.05] BOONDOX-b-calo}{}
\DeclareMathAlphabet{\mathlb}{U}{BOONDOX-calo}{m}{n}
\SetMathAlphabet{\mathlb}{bold}{U}{BOONDOX-calo}{b}{n}
\DeclareMathAlphabet{\mathblb}{U}{BOONDOX-calo}{b}{n}

\newcommand*\xbar[1]{%
  \hbox{%
    \vbox{%
      \hrule height 0.5pt % The actual bar
      \kern0.5ex%         % Distance between bar and symbol
      \hbox{%
        \kern-0.1em%      % Shortening on the left side
        \ensuremath{#1}%
        \kern-0.1em%      % Shortening on the right side
      }%
    }%
  }%
} 

\newtheorem{theorem}{Theorem}[section]
\newtheorem{lemma}[theorem]{Lemma}
\newtheorem{proposition}[theorem]{Proposition}
\newtheorem{Assumption}[theorem]{Assumption}
\newtheorem{algorithm}[theorem]{Algorithm}
\newtheorem{corollary}[theorem]{Corollary}
\newtheorem{condition}[theorem]{Condition}
\newtheorem{definition}[theorem]{Definition}

\usepackage[bottom]{footmisc}
\makeatletter
\renewcommand\@biblabel[1]{}
\renewenvironment{thebibliography}[1]
     {\section*{\refname}%
      \@mkboth{\MakeUppercase\refname}{\MakeUppercase\refname}%
      \list{}%
           {\leftmargin0pt
            \@openbib@code
            \usecounter{enumiv}}%
      \sloppy
      \clubpenalty4000
      \@clubpenalty \clubpenalty
      \widowpenalty4000%
      \sfcode`\.\@m}
     {\def\@noitemerr
       {\@latex@warning{Empty `thebibliography' environment}}%
      \endlist}
\makeatother

\makeatletter\@addtoreset{chapter}{part}\makeatother

\urlstyle{same}
 
\definecolor{darkblue}{rgb}{0.0,0.2,0.6}
\hypersetup{
    colorlinks=true,
    urlcolor=darkblue,
    citecolor=darkblue,
    linkcolor=black
}
 
\newcommand{\foo}{\makebox[0pt]{\textbullet}\hskip-0.5pt\vrule width 1pt\hspace{\labelsep}}
%\newcommand{\foo}{\hspace{-2.3pt}$\bullet$ \hspace{5pt}}
\newcommand{\hl}[1]{\textbf{\textcolor{red}{(#1)}}}
\newcommand{\sfix}[1]{{\typeout{SFIX on Page \thepage!}\setlength{\fboxrule}{1pt}\fcolorbox{black}{yellow}{{#1}}} }
\newcommand{\fix}[1]{\typeout{FIX on Page \thepage!} \standout{#1} }
\newcommand{\standout}[1]{ { 
    \renewcommand{\baselinestretch}{1.0} \normalsize \footnotesize
    \begin{center}
      \setlength{\fboxrule}{3pt}%
      \fcolorbox{black}{yellow}{
        \begin{minipage}{0.8\textwidth}
          {\bf \footnotesize #1}
        \end{minipage}
      }
    \end{center}
  } }

\makeatletter
\newenvironment{chapquote}[2][2em]
{\setlength{\@tempdima}{#1}%
  \def\chapquote@author{#2}%
  \parshape 1 \@tempdima \dimexpr\textwidth-2\@tempdima\relax%
  \itshape}
{\par\normalfont\hfill--\ \chapquote@author\hspace*{\@tempdima}\par\bigskip}
\makeatother

\graphicspath{{../graphs/}}
\renewcommand{\floatpagefraction}{0.1}
\begin{document}

This document presents a second working example for the Matlab toolbox from \citet{asv20}\footnote{Available for download under \url{https://github.com/adavernas/toolbox} }. We derive the model and relate the different expressions to the equations needed to solve the model numerically using the toolbox. \\
In this example, we present the model derived in \citet{dit16} where two agents have \citet*{epst89} utility functions and agents face a time-varying, idiosyncratic volatility. 

\paragraph{Technology}
Experts can use capital to produce consumption goods with this production technology:
\begin{equation}
y_t^i = \left[a _ \iota(g_t^i)\right]k_t^i
\end{equation}
where $g^i$ is the growth rate of the capital stock of agent $i$. In order to reach such as growth rate, agent $i$ must invest $\iota(g_t^i)$ consumption goods. Agent $i$'s capital stock follows an Ito process:
\begin{equation}
\frac{dk_t^i}{k_t^i} = g_t^idt + \sigma dZ_t + v_tdW_t^i
\end{equation} 
where $Z$ is an aggregate Brownian motion, and $W$ is an idiosyncratic Brownian motion of expert $i$.  The exposure to the aggregate shock $\sigma$ is assumed to be constant, while the load on the idiosyncratic shock is time-varying and follows an exogenous stochastic process of the form:
\begin{equation}
dv_t = \lambda(\bar{v} - v_t)dt + \sigma^v\sqrt{v_t}dZ_t
\end{equation}
The idiosyncratic shock cancel out in aggregate. The law of motion of aggregate capital $K_t  = \int_{[0,1]}k_t^idi$ is given by:
\begin{align*}
	dK_t = \left(\int_{[0,1]}k_t^idi\right)dt + \sigma K_tdZ_t
\end{align*}
\paragraph{Preferences} There are two types of agents: households $h \in H$ and experts $i \in I$. Both agents have stochastic differential utility, as developed by \citet*{duff92}. The utility of agent $j$ over his consumption process $c^j_t$ is defined as
\begin{align*}
U^j_t = \mathbb{E}_t\left(\int_t^\infty f\left(c^j_s,U^j_s\right)ds \right).
\end{align*}
The function $f_j(c,u)$ is a normalized aggregator of consumption and continuation value in each period defined as
\begin{align*}
f(c,U) = \frac{1-\gamma}{1-1/\zeta}U\left[\left(\frac{c}{((1-\gamma)U)^{1/(1-\gamma)}}\right)^{1-1/\zeta}-\rho\right]
\end{align*}
where $\rho$ is the rate of time preference, $\gamma$ is the coefficient of relative risk aversion, and $\zeta$ determines the elasticity of intertemporal substitution. 

\paragraph{Markets} Experts can trade capital at a competitive market price $q_t$, which has the following law of motion:
\begin{equation}
\frac{dq_t}{q_t} = \mu_t^qdt + \sigma_t^qdZ_t
\end{equation}
Financial markets are complete with a stochastic discount factor $\eta_t$ that follows the process:
\begin{equation}
\frac{d\eta_t}{\eta_t} = - r_tdt + \pi_tdZ_t
\end{equation}
where $r_t$ is the risk-free rate and $\pi_t$ is the price of \textbf{aggregate} risk. 

\paragraph{Households}
Households cannot hold capital of a single firm, but invest in the complete financial market. They choose their optimal consumption $c^j_t$ and exposure to aggregate risk $\sigma_t^w$ in order to maximize discounted infinite life time expected utilities $U^j_t$. Taking the stochastic discount factor $\eta_t$ as given, at any time, they solve the following maximization problem:
\begin{align*}
&\max_{c^h\geq', \sigma^w} U(c^h)\\
\text{subject to }\frac{dn_t^h}{n_t^h} = &(r_t + \sigma_t^w\pi_t - \frak{c}^h_t)dt + \sigma_t^wdZ_t
\end{align*}
where $n^h_t$ is the wealth of a household, $ \mathfrak{c}^h_t  = c^h_t /n^h_t $ his consumption rate. $Z_t$ is a standard Brownian motion that reflects aggregate risk.

\paragraph{Expert's problem}
Experts can trade capital and use capital for production, as well as participate in the aggregate financial market. The cumulative return from investing in capital for an expert $i$ is given by $R_t^{k,i}$ that follows:
\begin{equation*}
dR_t^{k,i} = \underbrace{\vphantom{\frac{a}{b}} \left[ \frac{a - \iota(g_t^i)}{q_t} + g_t^i + \mu_t^q + \sigma\sigma_t^q\right]}_{r_t^{ki}}dt + (\sigma + \sigma_t^q)dZ_t + v_tdW_t^i
\end{equation*}
Thus, an expert chooses her consumption and trading strategies to maximize her lifetime expected utility
\begin{align*}
&\max_{c^i\geq0, g, k\geq0, \theta}U(c^i)\\
\text{subject to } \frac{dn_t^i}{n_t^i} = &(\mu_t^{n,i} - \frak{c}_t^i)dt + \sigma_t^{n,i}dZ_t + \tilde{\sigma}_t^{n,i}dW_t^i
\end{align*}
where 
\begin{align*}
&\mu_t^{n,i} = r_t + q_tk_t^i(r_t^{ki} - r_t) - (1 - \phi) q_tk_t^i(\sigma + \sigma_t^q)\pi_t + \theta_t^i\pi_t\\
&\sigma_t^{n,i} = \phi q_tk_t^i(\sigma + \sigma_t^q) + \theta_t^i\\
&\tilde{\sigma}_t^{n,i} = \phi q_tk_t^iv_t
\end{align*}

\paragraph{Solving the HJB}We will guess and verify that the homotheticity of preferences allows us to write the value function for agents of type $j$ as: 
\begin{align}\label{eq:guess}
U\left(n^j_t,\xi^j_t\right) = \frac{\left(n^j_t\xi^j_t\right)^{1-\gamma}}{1-\gamma},
\end{align}
where variable $\xi^j_t$ follows 
\begin{align*}
\frac{d \xi^j_t}{\xi^j_t} = \mu^{\xi,j}_t dt + \sigma^{\xi,j}_t dZ_t
\end{align*}
Using the guess of the value function, the Hamilton-Jacobi-Bellman (HJB) equation for experts $i$ is equal to
\begin{align*}
	0 = &\max \Bigg\{ \Big(  \frac{\rho}{1-1/ \varpi} \Big) \Big[ \Big( \frac{c_t^j}{\xi_t^j} \Big)^{1-1/\varpi} - 1 \Big]\\
        &\quad+ \mu^{j,\xi}_t +  \mu_t^{nj} - \frac{\gamma}{2} (\sigma^{nj}_t )^2  - \frac{\gamma}{2} (\sigma^{\xi j}_t)^2 \\
        &\quad+ (1-\gamma) \sigma^{\xi j}_t \sigma^{nj}_t   - \frac{\gamma}{2}(\tilde{\sigma}_t^j)^2 \Bigg\}
\end{align*}
while households face the following HJB:
\begin{align*}
	0 = &\max\Bigg\{ \Big(  \frac{\rho}{1-1/ \varpi} \Big) \Big[ \Big( \frac{c_t^j}{ \xi_t^j} \Big)^{1-1/\varpi} - 1 \Big]\\
        &\quad+ \mu^{j,\xi}_t +  \mu_t^{nj} - \frac{\gamma}{2} (\sigma^{nj}_t )^2  - \frac{\gamma}{2} (\sigma^{\xi j}_t)^2 \\
        &\quad+ (1-\gamma) \sigma^{\xi j}_t \sigma^{nj}_t\Bigg\}
\end{align*}
The first order conditions with respect to consumption give:
\begin{align}
\frak{c}_t^i &= \rho^{1/\psi}(\xi_t^i)^{(\psi - 1)/\psi}\\
\frak{c}_t^h &= \rho^{1/\psi}(\xi_t^h)^{(\psi - 1)/\psi}\\
\end{align}
The first order condition for $g$ is equal to 
\begin{align}
\iota'(g_t^i) = q_t
\end{align}
From the first order condition for $\theta$, we get
\begin{align}
	\sigma_t^{ni} = \frac{\pi_t}{\gamma} - \frac{\gamma - 1}{\gamma}\sigma_t^{\xi,i}
\end{align}
while the first order condition for $\sigma_t^{nh}$ from the households' maximization problem yields
\begin{align}
	\sigma_t^{nh} = \frac{\pi_t}{\gamma} - \frac{\gamma - 1}{\gamma}\sigma_t^{\xi,h}
\end{align}
\paragraph{Optimality Conditions} The first order conditions with respect to $\frak{c}^j_t,\iota^j_t$, and $w^j_t$ are given by
\begin{align*}
\left(\frak{c}^j_t\right)^{-1/\zeta} = \left(\xi^j_t\right)^{\frac{1-1/\zeta}{1-\gamma}},
\end{align*}
\begin{align*}
1/q_t = \Phi_{\iota}(\iota_t),
\end{align*}
\begin{align*}
\mu^{r,j}_{t}-r_t - \gamma w^j_t \big(  \sigma^{q,\sigma}_t \big)^2  - \gamma w^j_t   \big( \sigma_t + \sigma^{q,k}_t \big)^2  +   \sigma^{q,\sigma}_t \sigma^{\xi,\sigma,j}_t  + \big( \sigma_t + \sigma^{q,k}_t\big) \sigma^{\xi,k,j}_t =0.
\end{align*}

Plugging in the optimality conditions in the HJB gives:
\begin{align}\label{eq:opt}
0 = & \frac{1}{1-1/\zeta} \left(\frak{c}^j_t-\rho\right) + r_t - \frak{c}^j_t + \frac{\gamma}{2} \big(  w^j_t  \sigma^{q,\sigma}_t \big)^2  + \frac{\gamma}{2}   \big( w^j_t  \sigma_t + w^j_t \sigma^{q,k}_t \big)^2 + \frac{\mu^{\xi,j}}{1-\gamma}. 
\end{align}

\paragraph{Market Clearing Conditions} We start by providing the definition of such an equilibrium in the state variables $\{x_t, v_t\}$, where $x_t$ is defined as the share of wealth in the hands of the intermediaries:
\begin{align*}
x_t  = \frac{n^i_t}{n^h_t+n^i_t} = \frac{n^i_t}{q_t k_t}.
\end{align*}
Then, we can use the market clearing condition for consumption to find $q_t$. Market clearing for consumption dictates that consumption from both types of agents equals the surplus from the production technology. So we have: 
\begin{align*}
\left(\frak{c}_t^ix_t + \frak{c}^h(1 - x_t)\right)q_t = a - \iota(g^i_t)
\end{align*}
The market for capital clears:
\begin{align}
	q_tk_t^ix_t = 1
\end{align}
And financial markets clear:
\begin{align}
	\sigma_t^{ni}x_t + \sigma_t^{nh}(1 - x_t) = \sigma + \sigma_t^q
\end{align}
Now let 
\begin{equation}
\psi_t \equiv \frac{w^i_t n^i_t}{w^i_t n^i_t + w^h_t n^h_t} = w^i_t \eta_t,
\end{equation} 
So finally, we have
\begin{equation}
\left(\frak{c}^i_t  \eta_t+  \frak{c}^h_t (1-\eta_t) \right) q_t= \psi_t (a^i - \iota_t) + (1-\psi_t) (a^h -  \iota_t)
\end{equation} 
The market clearing condition for capital allows us to identify $r_t$:
\begin{align*}
&k_t^i + k_t^h = k_t\\
&\frac{k_t^i}{k_t} + \frac{k_t^h}{k_t} = 1\\
&\frac{k_t^in_t^iq_t}{n_t^iq_tk_t} + \frac{k_t^hn_t^hq_t}{n_t^hq_tk_t} = 1\\
&w^i_t  \eta_t +  w^h_t (1-\eta_t) = 1.
\end{align*} 

Further, using our definition of $x_t$ and Ito's lemma, we can derive the law of motion of $x_t$ as: 
\begin{equation}\label{eq:lometa}
\begin{split}
\frac{dx_t}{x_t}=  &\bigg(\mu_t^{ni} - \frak{c}_t^i - \mu_t^q - g_t^i - \sigma_t^q\sigma  + (\sigma + \sigma_t^q)^2 - \sigma_t^{ni}(\sigma + \sigma_t^q) \bigg) dt \nonumber\\
& + \big(\sigma_t^{ni} + \sigma + \sigma_t^q\big)dZ_t
\end{split}
\end{equation}


\paragraph{Linking the model to the code} We start by collecting parameters and variables.
The model parameters are reported in table \ref{tab:para}. The parameters have to be specified (as well as given a value) in the section \textit{Parameters} in the file \texttt{model.m} of the toolbox.
\begin{table}
\center
\caption{Model parameters}
\begin{tabular}{ll} \toprule
Parameter & Definition \\ \midrule
$\gamma$ & relative risk aversion\\
$\psi$ & intertemporal elasticity of substitution\\
$\rho$  & discount rate\\ 
$\delta$ & \\ \midrule
$a^i$ & productivity of agent $i$\\
$a^h$ & productivity of agent $h$\\ 
$\kappa_p$ & investment costs \\ \midrule
$\kappa_z$ & drift of volatility \\
$\bar{v}$ & average volatility \\
$\varsigma$ & loading of volatility process \\
$\sigma$ & \\
$\phi$ & \\
$\tau$ & \\
$\lambda$ & \\
$A$ & \\
$B$ & \\ \toprule
\label{tab:para}
\end{tabular}
\end{table}
Next, we focus on the variables. Model specific variables are shown in table \ref{tab:end}. Endogenous variables are specified in the array \texttt{vars} in section \textit{Variables}  in the file \texttt{model.m} of the toolbox, while secondary variables are listed in \texttt{vars2}.
\begin{table}
\center
\caption{Variables}
\begin{tabular}{ll} \toprule
Variables & Definition \\ \midrule
Endogenous & $q_t$, $\mu_t^{x}$, $\sigma_t^{nh}$, $\sigma_t^{x}$, $\theta_t$\\  \midrule
Secondary    & $\frak{c}_t^i$, $\frak{c}_t^h$, $x$, $v$,\\
                     & $\tilde{\sigma}_t^z$, $\mu_t^{q}$, $r_t$, $\mu_t^{ni}$,$\mu_t^{nh}$, $\sigma_t^{ni}$, \\
                     & $\sigma_t^{\xi i}$, $\sigma_t^{\xi h}$, $g_t^i$, $\mu_t^v$, $\pi_t$, $\iota^i$, $\sigma_t^{q}$ \\ \toprule 
\label{tab:var}
\end{tabular}
\end{table}
The secondary variables are defined as follows. The two state variables are 
\begin{align}
&x_t = e\\
&v_t = z
\end{align}
In the code, the wealth multipliers are 
\begin{align}
&\xi_t^i = vi\\
&\xi_t^h = vh\\
\end{align}

Consumption-to-wealth ratio is given by the first order condition:
\begin{align}
&c_t^i = \rho^{1/\psi}(\xi_t^i)^{\frac{\psi - 1}{\psi}}\\
&c_t^h =\rho^{1/\psi}(\xi_t^h)^{\frac{\psi - 1}{\psi}}
\end{align}
From the market clearing conditions, we have:
\begin{align}
 k = \frac{1}{q_tx_t}
\end{align}

Investment maximization problem yields
\begin{align}
&\iota_t^i = \frac(1){2A}(q_t - B) - \delta
\end{align}
By assumption, we have:
\begin{align}
	&\mu_t^v = \kappa_z(\bar{v} - v)
\end{align}
The functional form for $\Phi^j$ was assumed to be:
\begin{align}
&\Phi^i = log(1 + \kappa_p\iota_t^i)/\kappa_p - \delta^i\\
&\Phi^h = log(1 + \kappa_p\iota_t^h)/\kappa_p - \delta^h
\end{align}
The drift of the state variable $\sigma_t$ was assumed to be
\begin{align}
&\mu^{\sigma} = \kappa(\sigma_t - \bar{v_t})\\
&\tilde{\sigma}_t^v = \sigma_t^v\sqrt{v_t}
\end{align}

Using Ito's lemma, we derive
\begin{align}
	\sigma_t^{\xi i} = \frac{\xi_{x}^i}{\xi^i}\sigma^x x +  \frac{\xi_{v}^i}{\xi^i}\tilde{\sigma}_t^v\\
	\sigma_t^{\xi h} = \frac{\xi_{x}^h}{\xi^h}\sigma^x x +  \frac{\xi_{v}^h}{\xi^h}\tilde{\sigma}_t^v	
\end{align}
Similarly, we have 
\begin{align}
	\sigma_t^{q} = &\frac{q_{x}}{q_t}\sigma_t^x x_t +  \frac{q_{v}}{q_t}\tilde{\sigma}_t^v\\
	\mu_t^q = &\frac{q_x}{q_t}\mu_t^x + \frac{q_v}{q_t}\mu_t^vv_t \\
			& + \frac{1}{2}\frac{q_{xx}}{q_t}(\sigma_t^x)^2\\
			& + \frac{1}{2}\frac{q_{vv}}{q_t}(\tilde{\sigma}_t^v)^2v_t \\
			& + \frac{q_{xv}}{q_t}\sigma_t^x\tilde{\sigma}_t^v\sqrt{v_t}
\end{align}
After some algebra using the experts' FOCs, we get an expression for the risk-free rate:
\begin{align}
	r_t = & (a - \iota_t^i)/q_t + g_t^i + \mu_t^q + \sigma\sigma_t^q - (1 - \phi)(\sigma + \sigma_t^q)\pi_t \\
	        & -\gamma(\sigma + \sigma_t^q)(\phi q_tk_t(\sigma + \sigma_t^q) + \theta) \\
	        & + ( 1 - \gamma)\phi(\sigma + \sigma_t^q)\sigma_t^{\xi i}\\
	        & - \gamma q_tk_t(\phi v)^2
\end{align} 
And from the two budget constraints, we have:
\begin{align}
	\mu_t^{ni} &= r_t + \gamma(\phi v)^2/x^2 + \pi_t\sigma_t^{ni}\\
	\mu_t^{nh} &= r_t  + \pi_t\sigma_t^{nh}
\end{align}
The model is closed be defining the endogenous variables:
\begin{align}
	&eqmux = (\mu_t^{ni} - c_t^i - \mu_t^q - g_t^i - \sigma_t^q\sigma + (\sigma + \sigma_t^q)^2 - \sigma_t^{ni}(\sigma + \sigma_t^q)x_t - \mu_t^v\\
	&eqq      = (c_t^ix_t + c_t^h(1 - x_t))q_t - (a - g_t^i) \\
	&eqsignh = \sigma_t^{ni}x_t + \sigma_t^{nh}(1 - x_t) - (\sigma + \sigma_t^q)\\
	&eqsigx   = (\sigma_t^{ni} + \sigma + \sigma_t^q)x_t - \sigma_t^x\\
	&eqtheta = \pi_t + ( 1 - \gamma)\sigma_t^{\xi i} - \gamma\phi q_tk_t(\sigma + \sigma_t^q) - \gamma\theta_t
\end{align}

\nobibliography{bibliography}

\end{document}
